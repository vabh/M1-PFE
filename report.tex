\documentclass[journal, a4paper, 12pt]{article}
%{IEEEtran}

% some very useful LaTeX packages include:

%\usepackage{cite}      % Written by Donald Arseneau
                        % V1.6 and later of IEEEtran pre-defines the format
                        % of the cite.sty package \cite{} output to follow
                        % that of IEEE. Loading the cite package will
                        % result in citation numbers being automatically
                        % sorted and properly "ranged". i.e.,
                        % [1], [9], [2], [7], [5], [6]
                        % (without using cite.sty)
                        % will become:
                        % [1], [2], [5]--[7], [9] (using cite.sty)
                        % cite.sty's \cite will automatically add leading
                        % space, if needed. Use cite.sty's noadjust option
                        % (cite.sty V3.8 and later) if you want to turn this
                        % off. cite.sty is already installed on most LaTeX
                        % systems. The latest version can be obtained at:
                        % http://www.ctan.org/tex-archive/macros/latex/contrib/supported/cite/

\usepackage{graphicx}   % Written by David Carlisle and Sebastian Rahtz
                        % Required if you want graphics, photos, etc.
                        % graphicx.sty is already installed on most LaTeX
                        % systems. The latest version and documentation can
                        % be obtained at:
                        % http://www.ctan.org/tex-archive/macros/latex/required/graphics/
                        % Another good source of documentation is "Using
                        % Imported Graphics in LaTeX2e" by Keith Reckdahl
                        % which can be found as esplatex.ps and epslatex.pdf
                        % at: http://www.ctan.org/tex-archive/info/

%\usepackage{psfrag}    % Written by Craig Barratt, Michael C. Grant,
                        % and David Carlisle
                        % This package allows you to substitute LaTeX
                        % commands for text in imported EPS graphic files.
                        % In this way, LaTeX symbols can be placed into
                        % graphics that have been generated by other
                        % applications. You must use latex->dvips->ps2pdf
                        % workflow (not direct pdf output from pdflatex) if
                        % you wish to use this capability because it works
                        % via some PostScript tricks. Alternatively, the
                        % graphics could be processed as separate files via
                        % psfrag and dvips, then converted to PDF for
                        % inclusion in the main file which uses pdflatex.
                        % Docs are in "The PSfrag System" by Michael C. Grant
                        % and David Carlisle. There is also some information
                        % about using psfrag in "Using Imported Graphics in
                        % LaTeX2e" by Keith Reckdahl which documents the
                        % graphicx package (see above). The psfrag package
                        % and documentation can be obtained at:
                        % http://www.ctan.org/tex-archive/macros/latex/contrib/supported/psfrag/

%\usepackage{subfigure} % Written by Steven Douglas Cochran
                        % This package makes it easy to put subfigures
                        % in your figures. i.e., "figure 1a and 1b"
                        % Docs are in "Using Imported Graphics in LaTeX2e"
                        % by Keith Reckdahl which also documents the graphicx
                        % package (see above). subfigure.sty is already
                        % installed on most LaTeX systems. The latest version
                        % and documentation can be obtained at:
                        % http://www.ctan.org/tex-archive/macros/latex/contrib/supported/subfigure/

\usepackage{url}        % Written by Donald Arseneau
                        % Provides better support for handling and breaking
                        % URLs. url.sty is already installed on most LaTeX
                        % systems. The latest version can be obtained at:
                        % http://www.ctan.org/tex-archive/macros/latex/contrib/other/misc/
                        % Read the url.sty source comments for usage information.

%\usepackage{stfloats}  % Written by Sigitas Tolusis
                        % Gives LaTeX2e the ability to do double column
                        % floats at the bottom of the page as well as the top.
                        % (e.g., "\begin{figure*}[!b]" is not normally
                        % possible in LaTeX2e). This is an invasive package
                        % which rewrites many portions of the LaTeX2e output
                        % routines. It may not work with other packages that
                        % modify the LaTeX2e output routine and/or with other
                        % versions of LaTeX. The latest version and
                        % documentation can be obtained at:
                        % http://www.ctan.org/tex-archive/macros/latex/contrib/supported/sttools/
                        % Documentation is contained in the stfloats.sty
                        % comments as well as in the presfull.pdf file.
                        % Do not use the stfloats baselinefloat ability as
                        % IEEE does not allow \baselineskip to stretch.
                        % Authors submitting work to the IEEE should note
                        % that IEEE rarely uses double column equations and
                        % that authors should try to avoid such use.
                        % Do not be tempted to use the cuted.sty or
                        % midfloat.sty package (by the same author) as IEEE
                        % does not format its papers in such ways.

\usepackage{amsmath}    % From the American Mathematical Society
                        % A popular package that provides many helpful commands
                        % for dealing with mathematics. Note that the AMSmath
                        % package sets \interdisplaylinepenalty to 10000 thus
                        % preventing page breaks from occurring within multiline
                        % equations. Use:
%\interdisplaylinepenalty=2500
                        % after loading amsmath to restore such page breaks
                        % as IEEEtran.cls normally does. amsmath.sty is already
                        % installed on most LaTeX systems. The latest version
                        % and documentation can be obtained at:
                        % http://www.ctan.org/tex-archive/macros/latex/required/amslatex/math/

\usepackage{subcaption}


% Other popular packages for formatting tables and equations include:

%\usepackage{array}
% Frank Mittelbach's and David Carlisle's array.sty which improves the
% LaTeX2e array and tabular environments to provide better appearances and
% additional user controls. array.sty is already installed on most systems.
% The latest version and documentation can be obtained at:
% http://www.ctan.org/tex-archive/macros/latex/required/tools/

% V1.6 of IEEEtran contains the IEEEeqnarray family of commands that can
% be used to generate multiline equations as well as matrices, tables, etc.

% Also of notable interest:
% Scott Pakin's eqparbox package for creating (automatically sized) equal
% width boxes. Available:
% http://www.ctan.org/tex-archive/macros/latex/contrib/supported/eqparbox/

% *** Do not adjust lengths that control margins, column widths, etc. ***
% *** Do not use packages that alter fonts (such as pslatex).         ***
% There should be no need to do such things with IEEEtran.cls V1.6 and later.

\setlength{\parindent}{2em}


% Your document starts here!
\begin{document}

% Define document title and author
      \title{How social links are created on Twitter, and what is their privacy implication?\\
    (Analysis of hashtag behavior (relations? ) to identify unusual user activities.)}
      \author{Anuvabh Dutt, Anastasia Kuznetsova \\ \textit{under the supervision of} Arnaud Legout
%     \thanks{Advisor: Arnaud Legout, Inria}
    }
    
%     \markboth{
%     tseminar Digitale Kommunikationssysteme}{}
      \maketitle

% Write abstract here
\begin{abstract}
      The short abstract (50-80 words) is intended to give the reader an overview of the work.
\end{abstract}

% Each section begins with a \section{title} command
\section{Introduction}
      % \PARstart{}{} creates a tall first letter for this first paragraph
% \PARstart{T}{witter} 
Twitter is an online social network in which registered users can read and send short messages of 140 characters, called tweets. As of March 2015, Twitter has 302 million active users~\cite{twitter_stats}. Twitter is a directed social network. The relations between users on the network are that of ‘followers’ and ‘followees’. This is different from other social networks like Facebook where relations between users are mutual. The central point of Twitter's business is the tweet. Consequently, Twitter wants to have a method to classify tweet content as either good or bad. Currently there are spam filters but they do not work with a 100\% accuracy.  We take an alternate approach to classifying tweets. We study the way content in a tweet is connected to other posts and use this relation for the classification, by trying to identify users who are likely to post spam content. \\
    
Twitter hosts a large number of influential people. There are many heads of states, scientists, actors, sportspeople who are active Twitter users. Twitter provides a very convenient platform for these people to connect with their followers. These users, by virtue of being well known, have a lot of followers and they can reach a large audience with less effort. Due to this they also have a certain level of influence over the network. Posts of these users have a very high probability to become ‘viral’ as compared to ordinary users. Tweets of popular users are tweeted again by their followers in a procedure called ‘retweeting’. The number of retweets of a tweet is a measure of the popularity of the post. Additionally hashtags \footnote{A hashtag is a type of label or metadata tag used on social network and microblogging services which makes it easier for users to find messages with a specific theme or content. ~\cite{wikihashtag}} used by such users are used by their followers who want to join in the conversation. For example during Roland Garros 2015, people who wanted to post about the event, included in their tweets, the hashtag \#RG15. Users who wanted to see what other people were posting about this event just had to search for this hashtag. These hashtags form a target for spam. Spammers can use such hashtags to piggyback their spam content which is usually unrelated to the original hashtag. Hashtags are the entity that users use to post about any given topic. As a result popular topics directly imply that the corresponding hashtags will be present in a large number of tweets and also these tweets will have a big audience. These are also called hashtag campaigns. In such campaigns, users are invited to include this specific hashtag in their tweets. This gives spammers a target hashtag for their campaigns. Spammers include a popular hashtag in their tweets to reach Twitter users. \\

In this work we study the influencing power of popular users and the role popular hashtags play in the context of spam activity in the network. The effect of the activity of popular users is important because they have a large influence. This influence is critical as they can manipulate the behavior of other users. In times of elections, a popular user has the power to align the thoughts of their followers with their own. If such an account is compromised, the intruder might cause misinformation to propagate. We measured this influence in terms of the activity of their followers after they post a tweet. Gaining a large number of followers for a spam acccount is unlikely and so a spammer has the other option of trying to spread spam content using hashtags. The hope of the spammer is that a user will search for a popular hashtag and as a result be exposed to the spam content. With regards to the hashtags and their role in the spreading of spam, we present a technique that can be used to weed out spam tweets from valid tweets, given a specific hashtag campaign. \\

In section 4, we describe the influence of popular users on the regular users in Twitter. To qualify as popular a user needs to have a large audience. In section 5, we discuss the concept of the hashtag graph and how it can be used to identify users with suspicious activity. In section 5.5, we present the details of our findings.


% Main Part
\section{Related Work}

Thomas \textit{et al.}~\cite{thomas_paper} describe the general characteristics of spam activity in Twitter. The paper describes the tools used by spammers and the how they carry out their activities. Our work uses this information to perform the experiments. The paper does not present any methods to deal with spam activity. Other work on spam activity on twitter has involved machine learning techniques and user relations. One example is the work of Benevenuto \textit{et al.}~\cite{benevenuto_paper}. Our work does not use such parameters. We focus on the tweet content itself and how each post is connected to the other posts in Twitter. We construct a graph and obtain our results by studying the properties of this graph. We believe that this reduces subjective factors and relies more on objective facts.  
% [insert twitter spam papers] [insert study popularity of users]

\section{Datasets}

We obtained our datasets by using the publicly available Twitter API. We used the REST API to collect tweets related to hashtags and search terms. The collection was done several times during the course of the project work. We used the Streaming API to collect a 1\% sample of tweets over five days (3/6/2015 - 8/6/2015). For this purpose we built a crawler and deployed in on our personal computers. Using the crawler we obtained tweets related to the search terms that we queried for. The public API is rate limited, meaning that we can query for a limited time and for a limited number of tweets. To get around this problem we used multiple accounts. Our keywords were chosen based on their popularity as either a hashtag or as a search term. \\

Number of tweets collected are:
\begin{itemize}
\item \#HappyPiDay - 18 543 tweets.
\item \#Hillary2016 - 159 582 tweets.
\item \#iPhone - 97 315 tweets.
\item 1\% sample of tweets in real time - 7 697 223 tweets over 5 days.
\end{itemize}

Further details of specific data sets come inside the other sections when we describe our analysis and the corresponding results.


\section{Influence of Popular Users}

Many users on twitter have a very large number of followers ( \textgreater 1 000 000). It is interesting to study the influence that these users have on their audience because they have the ability to reach a significant portion of the Twitter users and as such could have a pivotal role in information propagation in the Twitter social network. To study this behavior, we decide to observe their activities with respect to a global event. We chose the event \#HappyPiDay. This day was a global event to commemorate the number  $\pi$. If we represent the date as mm/dd/yy we get 3/14/15 on 3rd March 2015, which are the first few digits of $\pi$.
\\

For our analysis, we collected public tweets for 3rd March 2015. Figure~\ref{fig:piday_freq_tweets} shows the number of tweets posted per minute over a twenty four hour period. There are three peaks in the figure which mean that at those times the number of tweets being posted increased significantly.
They correspond to the activity of a popular user. At those times a popular user had posted a tweet and their followers posted retweets of the original tweet. This translated to a sudden increase in the number of tweets being posted. The red and green line correspond to the number of retweets of these accounts. We observe that they constitute only a small fraction of the total number of tweets posted. The details of these accounts are summarized in Table~\ref{tab:numberoftweets_piday}. 
\\

In Figure~\ref{fig:piday_freq_tweets}, the blue line shows the number of original tweets that are being posted. These tweets are not retweets. The original tweets form a peak between 13:00 and 14:00 GMT [9:00 and 10:00 ET]. This time is the ``$\pi$ time''. The timestamp at this instant is 3/14/15 9:26:23, corresponding to the first digits of $\pi$.
\\

Peaks are the result of activity of popular users or because of a global event. Popular users initiate an increase in tweets since they have a large number of followers. However, from Figure~\ref{fig:piday_freq_tweets} we observe that this activity is like an impulse. It lasts for a short time and does not significantly change user behaviour. In this dataset, the number of unique tweets stayed almost the same throughout the time period and was not affected much by the activities of the popular users.
\\
      
%     \begin{figure}[!hbt]
%           \begin{center}
%           \includegraphics[width=\columnwidth]{1_PidayTweetsFrequency.png}
%           \caption{Frequency of tweets posted by users per minute for \#HappyPIday}
%           \label{fig:piday_freq_tweets}
%           \end{center}
%     \end{figure}
    
    \begin{table}[!hbt]
            \begin{center}
            \caption{Different categories of tweets for \#HappyPiDay}
            \label{tab:numberoftweets_piday}
            \begin{tabular}{|p{4cm}|p{2cm}|p{2cm}|p{2cm}|}
                  \hline
                   & Number of tweets & Percent of number of tweets & Number of followers \\
            \hline
                  Total & 18 543 & 100\% & \\
                  \hline
                  Unique & 7 085  &  38,3\%  & \\
                  \hline
                  @TheTweetOfGod user retweets & 2 425 & 13\% & 1 900 000\\
            \hline
            @Cw-spn user retweets & 2 347 & 12,7\% & 1 010 000\\
                  \hline
            Other retweets & 6 686 & 36\% &\\
                  \hline
            \end{tabular}
            \end{center}
      \end{table}
    
    
    \begin{figure}[!hbt]
\begin{subfigure}{1\textwidth}
\includegraphics[width=\columnwidth]{1_PidayTweetsFrequency.png}
%\includegraphics[width=0.9\linewidth, height=5cm]{1_PidayTweetsFrequency.png} 
\caption{Frequency of tweets posted by users per minute for \#HappyPiDay.}
\label{fig:piday_freq_tweets}
\end{subfigure}
\begin{subfigure}{1\textwidth}
\includegraphics[width=\columnwidth]{2_Piday_Audience.png}
%\includegraphics[width=0.9\linewidth, height=5cm]{2_Piday_Audience.png}
\caption{Audience of tweets. Computed as a sum of number of followers for each user that posted a tweet for \#HappyPiDay.}
\label{fig:piday_audience}
\end{subfigure}
 
\caption{Representation of data collected for \#HappyPiDay in terms of the frequency of tweets and the size of the audience.}
\label{fig:piday_tweets_and_audience}
\end{figure}
   

Figure~\ref{fig:piday_audience} shows the audience of each posted tweet. The audience size is equal to the number of followers of the user who posts a tweet. We observe peaks in this figure as well. The peaks correspond to the number of followers of the popular users. We conclude that popular users have a very large audience size as compared to most users but their impact on the users posting tweets is not big. They do not significantly change the number of tweets being posted. 
\\
      
%     \begin{figure}[!hbt]
%           \begin{center}
%           \includegraphics[width=\columnwidth]{2_Piday_Audience.png}
%           \caption{Audience of tweets for \#HappyPIday. Computed as a sum of number of followers for each user that posted a tweet}
%           \label{fig:piday_audience}
%           \end{center}
%     \end{figure}
    
We extend our analysis to another hashtag, \#Hillary2016. \#HappyPiDay was a global event and was not started by any specific user. It became popular because of the collective behaviour of Twitter users. To asses the behavior of a tweet frequency from another perspective, we decided to choose a hastag created and made popular by a specific user with a large audience. We chose the hashtag \#Hillary2016 which was started by the official account of Hillary Clinton right after she announced that she is running for the President of the United States in 2016. The account @HillaryClinton has 3,64m followers,a significant audience for tweets posted by this account. In addition, since the topic is related to the Hillary Clinton announcement, it is expected to have a long life cycle, at least up till November 8, 2016, the day of the presidential election in USA. This gives us the possibility to collect data for a long period of time as compared to \#HappyPiDay because the activity of users and the number of tweets related to this topic does not depend on the specific date.
\\

Figure~\ref{fig:hillary_freq_tweets} represents the number of tweets posted with the \#Hillary2016 per hour. Data was collected for the following time periods:
\begin{itemize}
\item 13th April - 21th April
\item 3rd May - 11th May
\item 25th May - 3rd June
\end{itemize}

Total collected tweets is 159582. We see a peak on the 13th of April, when the hashtag \#Hillary2016 was created. After 14th of April we observe a periodical behavior with a decreasing trend till the 5th of May. The periodicity is due to the time of a day. Four peaks between May 5th ans May 10th are due to the external events and news about Hillary Clinton appeared in this period which caused discussions in social networks. The general trend for the \#Hillary2016 is decreasing, from 24380 tweets on 13th of April to 2193 tweets on 30th of May with the average of 5910 tweets per day. 

    
      \begin{figure}[!hbt]
            \begin{center}
            \includegraphics[width=\columnwidth]{3_HillaryTweetsFrequency_with_retweets.png}
            \caption{Frequency of tweets posted by users per minute for \#Hillary2016.}
            \label{fig:hillary_freq_tweets}
            \end{center}
      \end{figure}
    
To conduct the same analysis as for the \#HappyPiDay, we calculated the same metrics that are presented in the Table~\ref{tab:numberoftweets_hillary}. As we can see, the percent of unique tweets in both datasets if very close, 38,2\% for \#HappyPiDay and 32,4\% for \#Hillary2016. GreeBut in the case with \#Hillary2016 top 2 retweeted tweets only take 7,7\% of all posted tweets, their impact on the tweet frequency for this time period is lower because of size of the dataset. On average each tweet is retweeted 9 times, that gives us the big number of retweets in general, without those retweets, that could significantly impact the overall frequency of tweets. This supports the assumption that number of tweets posted for a topic does not depend a lot on the activity of popular users.
      
    \begin{table}[!hbt]
            \begin{center}
            \caption{Different categories of tweets for \#Hillary2016}
            \label{tab:numberoftweets_hillary}
            \begin{tabular}{|p{4cm}|p{2cm}|p{2cm}|p{2cm}|}
                  \hline
                   & Number of tweets & Percent of number of tweets & Number of followers \\
            \hline
                  Total & 159 582 & 100\% & \\
                  \hline
                  Unique & 51 664  &  32,4\%  & \\
                  \hline
                  @JessicaShores user retweets & 10 181 & 6,4\% & 18 000\\
            \hline
            @ComplexPop user retweets & 1 444 & 0,9\% & 19 300\\
                  \hline
            Other retweets & 96 293 & 60,3\% &\\
                  \hline
            \end{tabular}
            \end{center}
      \end{table}
    

% In the dataset we encountered a significant number of hashtag campaigns as a single tweet with a set of hashtags that was tweeted or retweeted many times by different users. Some of these campaigns were in support of Hillary Clinton and some were against her campaign. We observed that these tweets tend to have the specific set of hashtags: the “starter” hashtag \#Hillary2016 and a number of different hashtags, sometimes completely unrelated to the topic of the “starter” hashtag. \\

    
\section{The Hashtag Graph }

In this section we describe hashtags, the entity used in a social network to coordinate communication relating to a specific topic. We then describe the Hashtag Graph that we use to represent the connectedness among hashtags in different tweets. We also define the metrics that we use in our analysis.

\subsection{Hashtags}
In a social network, a hashtag serves as a tag for coordinating communication. It is  a word prefixed with the ‘\#’ symbol. A hashtag helps in content discovery and aids people to find all information related to a specific topic. In Twitter a user has the ability to search for a hashtag and view all tweets that contain that hashtag. The hashtag can serve as a weapon for spammers. If a hashtag becomes popular, there will be a large number of tweets that contain that specific hashtag. Spammers can include the hashtag in their tweets and be sure that people will view it. This serves as a convenient medium for spreading spam since the spammer does not even have to rely on the social links of users in the network. All that is required to do is identify popular hashtags and use those. Twitter lists the most popular hashtags at any given time. This aids the discovery of popular terms which are likely to be used in hashtags.\\
% describe evolution 

      \begin{figure}[!hbt]
            \begin{center}
            \includegraphics[width=\columnwidth]{hashtagevolution_newtags.png}
            \caption{Number of new hashtags observed in data collected for 5 days.}
            \label{fig:hashtagevolution}
            \end{center}
      \end{figure}

Figure~\ref{fig:hashtagevolution} shows the growth of hashtags in the Twitter network over a period of five days. The bumps correspond to diurnal changes. We observe that during the period of each day, for a certain time interval the rate at which new hashtags are introduced increases and then decreases to a previous value. On average there are approximately 74000 new hashtags every day. Out of these new hashtags some will become popular and in turn be potential targets for spammers. This increases the importance of studying the behavior of these hasthtags the role they play in social networks. \\

Multiple hashtags can occur in a tweet to give a sense of the subjects to which a tweet might relate to. We get a notion of connectedness among hashtags. To study this property we focused on the previously mentioned datasets for \#HappyPiDay and \#Hillary2016.


\subsection{The Graph}
We introduce the concept of the hashtag graph. We construct this graph from the hashtags present in each of the tweets of our dataset. A node in the graph corresponds to a hashtag. An edge is present between two nodes if those hashtags occur in the same tweet. Additionally we define the following metrics:
\begin{itemize}
\item
\textbf{Node weight or frequency} to be the number of times the hashtag occurs in the dataset.
\item
\textbf{Node degree} is defined as the number of different nodes a particular node is connected to. We can interpret this as a measure of the different hashtags a particular hashtag appears with in tweets.
\item
\textbf{Edge weight} is defined as the number of times two hashtags occur together. This gives an idea about the affinity of one hashtag with the other and conveys how often both occur together.
\end{itemize}

As an example of the hashtag graph that could be created, we took 5 tweets from the dataset, collected for \#HappyPiDay to create the graph. Figure~\ref{fig:tweets_piday} shows tweets as they appear in the stream after the search for \#HappyPiDay hashtag.
\\

  \begin{figure}[!hbt]
            \begin{center}
            \includegraphics[width=120mm]{Tweets_piday.png}
            \caption{A subset of 5 tweets from dataset collected for the \#HappyPiDay.}
            \label{fig:tweets_piday}
            \end{center}
      \end{figure}
    
From these tweets we extract the set of hashtags. The collected set of hashtags is:

\begin{enumerate}
\item \#TakeBackYourData, \#SXSW, \#Gratitude, \#HappyPiDay
\item \#HappyPiDay
\item \#Supernatural, \#HappyPiDay
\item \#HappyPiDay, \#IncomeInequality
\item \#HappyPiDay, \#Supernatural
\end{enumerate}


Each hashtag represents a distinct node of a graph. First time two hashtags appear together in the same tweet, we create an edge between them. Each time we observe same hashtags present together in the new tweet, we increment the weight of an edge between them by one.
\\

Figure~\ref{fig:tweets_graph_piday.png} represents the hashtag graph created from chosen tweets. Since \#HappyPiDay was the initial hashtag that we were collecting data for, it appears in all of the tweets and is connected to all of the other hashtags. Two hashtags \#Supernatural and \#IncomeInequality are only connected to \#HappyPiDay because there are no other hashtags in tweets with them. Along with the \#HappyPiDay, three hashtags (\#TakeBackYourData, \#SXSW, \#Gratitude) form a clique of hashtags, because they are present in the same tweet and are connected with each other. The notion of the clique of hashtags is described in the following section~\ref{cluques_subsection}.

    
        \begin{figure}[!hbt]
            \begin{center}
            \includegraphics[width=80mm]{tweets_graph_piday.png}
            \caption{Hashtag graph, represents the connections between hashtags in the collected dataset for \#HappyPiDay.}
            \label{fig:tweets_graph_piday.png}
            \end{center}
      \end{figure}

\subsection{Cliques and user activity}\label{cluques_subsection}
In the hashtag graph there are cliques formed by hashtags. An alternate representation with the same meaning is a set of hashtags which appear together in a tweet. An example of a clique of hashtags is shown on the Figure~\ref{fig:tweets_graph_piday.png}. The set of four hashtags (\#HappyPiDay, \#TakeBackYourData, \#SXSW, \#Gratitude) form a clique because all these hashtags appear in one tweet. An interesting result that we obtained is that users which were tweeting using the same set of hashtags formed a network amongst themselves. They had the same activity as shown in Figure~\ref{fig:spam_user_times}. We selected this specific set of hashtags manually. For future work we want to develop a classifier to select the hashtag set.
\\

Figure~\ref{fig:spam_user_times} shows the time when an account posted a tweet. Each dot corresponds to the time a tweet was posted. In the figure we see that all the accounts stopped tweeting at the same time. Moreover their posts follow a similar pattern. The time interval when some of the accounts do not post any tweet overlap completely. Such behaviour shows strong correlation between the tweet times of these accounts. These accounts are probably part of a botnet. On inspecting some these accounts, we observe that all of these accounts tweet using a set of hashtags, out of which one is a popular search term like:
\begin{itemize}
\item android
\item video
\item ipad
\end{itemize}

We discover that from the spammers perspective it is not good to have all of the accounts use the same set of hashtags. They can be discovered easily with our method. 
%Are you sure it is the good thing to write?

\begin{figure}[!hbt]
            \begin{center}
            \includegraphics[width=\columnwidth]{spam_user_times.png}
            \caption{Posting time of tweets by suspected spammers.}
            \label{fig:spam_user_times}
            \end{center}
      \end{figure}

\subsection{Analysis of the graph}

After creating the graph we computed the frequency of nodes, the degree of nodes and the weights of edges of the graph. We do this for \#HappyPiDay and \#Hillary2016.
\\
%and \#iPhone. \\ not sure if we should add it since we don't show cdfs for iphone

        \begin{figure}[!hbt]
            \begin{center}
            \includegraphics[width=\columnwidth]{Cdf_Hillary_Piday_Frequency.png}
            \caption{CDF of hashtag frequency \#HappyPiDay and \#Hillary2016.}
            \label{fig:cdf_hillary_piday_frequency}
            \end{center}
      \end{figure}
    
Figure~\ref{fig:cdf_hillary_piday_frequency} represents the cumulative distribution function of the frequency of hashtags. For all the campaigns, 50\% of hashtags appear once in the entire set of tweets. The outlier has the highest frequency as it is the hashtag for which we had crawled the data. \\

      \begin{figure}[!hbt]
            \begin{center}
            \includegraphics[width=\columnwidth]{Cdf_Hillary_Piday_Edge_Weight.png}
            \caption{CDF of weights of edges for \#HappyPiDay and \#Hillary2016.}
            \label{fig:cdf_hillary_piday_edge_weight}
            \end{center}
      \end{figure}
    
For the \#HappyPiDay campaign, the weights of edges (number of times two hashtags appear together in one tweet) are equal to 1 for about 70\% of edges. This means that 70\% of nodes are leaves and are only connected to the \#HappyPiDay hashtag. In this case we conclude that hashtags in \#Hillary2016 data set are more connected. To support this we have calculated the degree of each hashtag as a sum of weights of all adjacent edges. \\
    
        \begin{figure}[!hbt]
            \begin{center}
            \includegraphics[width=\columnwidth]{Cdf_Hillary_Piday_Node_weighted_Degree.png}
            \caption{CDF of weighted node degree for \#HappyPiDay and \#Hillary2016.}
            \label{fig:cdf_hillary_piday_weight_sum}
            \end{center}
      \end{figure}

We conclude that even if hashtags in the \#HappyPiDay data set are connected to a smaller number of hashtags, the degrees of nodes in both data sets are distributed similarly and depend mostly on the size of dataset and the frequency of a hashtag. \\

\subsubsection{Frequency and degree dependence}
\textbf{Frequency} of a hashtag is an important metric that allows us to understand if the hashtag is popular (if it has a high frequency). In this case its audience is wide and it allows to influence users that see tweets with this hashtag. If more users see the tweet with some advertisement, the click-throught rate \footnote{Click-throught rate - number of times the user clicked on the link divided by the number of times the link was shown.} of links embedded in the tweet will increase. This increases the value of the marketing or advertising campaigns as well as the value of spam campaigns. Hence, studying the behavior of a hashtag frequency is critical for understanding the behavior of spammers or organized advertisement campaigns on Twitter.\\

Another important metric is the \textbf{degree} of a hashtag: the number of different hashtags with which it appears in the same tweet. This metric shows if the hashtag is related to a different number of topics. We observed that popular hashtags related to some topic have high frequency and degree. They are used in different tweet and with a big number of other hashtags. \\

In the Table~\ref{tab:graph_freq_degree} we show an example of calculated frequency and degree for a subset of tweets extracted from the \#HappyPiDay dataset. The content of the tweets is shown on the Figure~\ref{fig:tweets_piday}. 

      \begin{table}[!hbt]
            \begin{center}
            \caption{Frequency and degree for the subset of tweets from the dataset collected for \#HappyPiDay.}
            \label{tab:graph_freq_degree}
            \begin{tabular}{|p{4cm}|p{2cm}|p{2cm}|}
                  \hline
                  Hashtag & Frequency & Degree \\
            \hline
                  1. HappyPIday & 5 & 5 \\
                  \hline
                  2. Supernatural & 2  &  1  \\
                  \hline
                  3. Gratitude & 1 & 3 \\
            \hline
            4. SXSW & 1 & 3 \\
                  \hline
            5. TakeBackYourData & 1 & 3\\
                  \hline
            6. IncomeInquality & 1 & 1 \\
                  \hline
            \end{tabular}
            \end{center}
      \end{table}

There are different characteristics of hashtags depending on the above mentioned metrics:

\begin{itemize}

\item Popular hashtags have a high frequency and a high degree.

\item Hashtags of medium popularity have moderate values for frequency and degree. They are related to the same topic but are used by a smaller amount of users.

\item Custom user hashtags which are very specific to a topic and are used by a very small fraction of users.

\item Hashtags not related to the initial one and having high frequency and small degree. They are connected to a small set of hashtags.
\end{itemize}

The last type of hashtag is an indication that they are being used by users with suspicious activity. \\

%    \begin{figure}[!hbt]
%           \begin{center}
%           \includegraphics[width=\columnwidth]{Scatter_Hillary_Piday_With_Retweets_Sum.png}
%           \caption{Scatterplot of weighted node degree and frequency of hashtags}
%           \label{fig:scatter_hillary_piday_with_retweets_sum}
%           \end{center}
%     \end{figure}


%To study dependence between frequency and degree at the beginning we decided to compute the weighted degree of a node as a sum of weights of edges adjacent to a node. In other words, the weighted degree of a node is a number of other hashtags the analyzed hashtag appears with multiplied by a number of tweets in which these hashtags appear together. After receiving results that are presented on the Figure~\ref{fig:scatter_hillary_piday_with_retweets_sum}, we observed high dependence between the frequency of a hashtag and its weighted degree. We were expecting to see outliers with the high frequency and low degree, which were present in the dataset. There results do not provide a suitable basis for further analysis, because the number of tweets in which two hashtags appear together will grow with the frequency. Hence, the weighted degree of a node will also grow according to the growth of frequency. 
%For a more relevant result, we chose a degree of a node as a number of adjacent nodes, without taking into account the weight of an edge between these two nodes. The result of this computation is presented on the Figure~\ref{fig:scatter_hillary_piday_with_retweets}. We observe a bigger amount of outliers, on which we focused our analysis.



        \begin{figure}[!hbt]
            \begin{center}
            \includegraphics[width=\columnwidth]{Scatter_Hillary_Piday_With_Retweets.png}
            \caption{Scatterplot of degree and frequency of hashtags.}
            \label{fig:scatter_hillary_piday_with_retweets}
            \end{center}
      \end{figure}
    

In Figure~\ref{fig:scatter_hillary_piday_with_retweets} we see that there is a correlation between frequency and degree. We see that there is a correlation between frequency and degree in terms of tweets limitations. If a hashtag has a low frequency it can't be connected to a significant number of other hashtags, because of the size limitations of a tweet (maximum of 140 characters in a tweet). Hence, there is an upper bound for the degree of a hahstag, based on the frequency. For both \#HappyPiDay and \#Hillary2016 we have encountered a set of nodes with high frequency and low degree, that is hashtags that appear a lot of times but are connected few other hashtags. These outliers are of interest. To study them, we computed the ratio between the frequency and the degree of hashtags. % In the set of hashtags with high frequency and low degree we observed a number of retweets which are not related to spam. These are not of interest. 
Retweets will have high frequency since they are posted by many users and will not have a high degree since they are essentially copies of the original tweet. We do further computations after removing retweets from our dataset. \\


    
        \begin{figure}[!hbt]
            \begin{center}
            \includegraphics[width=\columnwidth]{Ratio_Piday_Hillary_With_Retweets.png}
            \caption{Ratio between frequency and degree of a node for \#HappyPiDay and \#Hillary2016. Ratio is represented with a dot for each of the hashtags in the data set. Hashtags are sorted alphabetically.}
            \label{fig:ratio_hillary_piday_with_retweets}
            \end{center}
      \end{figure}
    

        \begin{figure}[!hbt]
            \begin{center}
            \includegraphics[width=\columnwidth]{Ratio_Piday_Hillary_Without_Retweets.png}
            \caption{Ratio between frequency and degree of a node for \#HappyPiDay and \#Hillary2016. Retweets are excluded.}
            \label{fig:ratio_hillary_piday_without_retweets}
            \end{center}
      \end{figure}
    
    
For the following analysis we have chosen the top hashtags for which frequency is at least five times bigger than the degree. This means that the number of times the hashtag  appears in a tweet is at least five times the number of different hashtags it is connected to. Table~\ref{tab:hashtagratio} contains information about a set of hashtags, that were chosen based on the following criteria:
\begin{itemize}
\item The hashtag is not related to the central topic.
\item The enclosing tweet of the hashtag has an obvious advertising or promotional goal.
\end{itemize}

  \begin{table}[!hbt]
            \begin{center}
            \caption{Characteristics of hashtags with degree to frequency ratio $>=$ 5.}
            \label{tab:hashtagratio}
            \begin{tabular}{|p{5cm}|p{3cm}|p{3cm}|}
                  \hline
                   & \#HappyPiDay & \#Hillary2016 \\
            \hline
                  Total number of hashtags & 2926 & 9519 \\
                  \hline
                  Number of hashtags with ratio $>=$ 5 & 18  &  104 \\
                  \hline
                  \% of hashtags with ratio $>=$ 5 & 0,6\% & 1,1\%\\
            \hline
            Number of suspicious hashtags & 7 & 23 \\
                  \hline
            \% of suspicious hashtags & 38,8\% & 22,1\%\\
                  \hline
            \end{tabular}
            \end{center}
      \end{table}

\subsection{Results and discussion}
In table~\ref{tab:hashtagratio}, we see that the overall number of hasgtags with a significant difference between the frequency and degree does not exceed 2\% of the total number of hashtags. This is an outcome of previously observed high number of hashtags which have degree $<=$ 1; the fact that the hashtag degree can't be significantly bigger than the frequency and that popular hashtags have degree and frequency values close to each other. Out of the number of hashtags with the ratio $>=$ 5, 38.8\%  and 22.1\% of hashtags for \#HappyPiDay and \#Hillary2016 were suspicious. The following Figures~\ref{fig:word_piday} and~\ref{fig:word_hillary} show the "word cloud" of the chosen top hashtags with the high ratio, where the size of the word correspond to the ratio value.
\\
        
     \begin{figure}[!hbt]
            \begin{center}
            \includegraphics[width=\columnwidth]{piday_ratio_top5.png}
            \caption{Word cloud of top hashtags with ratio bigger than or equal to 5 in \#HappyPiDay dataset.}
            \label{fig:word_piday}
            \end{center}
      \end{figure}
    
In figure~\ref{fig:word_piday} we see the \#str8n8v4lyf hashtag with a very high ratio equal to 48. Frequency and degree for this hashtag are equal to 48 and 1 respectively. This hashtag was posted 48 times by different users and is connected only to the \#HappyPiDay hashtag. After the inspection of accounts of these users we found that all of them have a usual tweeting behavior, none of them uses an automated way to post tweets or uses the \#str8n8v4lyf hashtag for promotion or spam. The second hashtag with the highest ratio is \#audio. It has frequency equal to 12, degree equal to one, and it was posted by a single user @audio\_\_agent. This user has 1616 followers, 1968 followings and 136000 tweets. All the tweets are posted with a very high frequency and in a similar manner: they usually have a short message, set of hashtags and an embedded link, which leads to a web-site with advertisement content~\cite{dragplus_audio}. We can identify this user as a spammer. \\

Next suspicious hashtag with ratio equal to 7, frequency equal to 7 and degree equal to 1 is \#saturdaynightonline. This hashtag was started by six users: @DMarainWUNF, @WUNFMorningShow, @KyleMazzaWUNF, @WUNFNewsDesk, @WUNFNews and @SaturdayOnline. First three accounts belong to the journalists, reporters of UNF News radio channel. @WUNFNewsDesk and @WUNFNews are the official accounts of this radio channel. The last one is the official account of the American radio show "Saturday Night On-line".  All of these accounts have a high frequency of tweets but the tweeting behavior differs. There is no particular pattern that could identify the behavior of this set of users. \#HappyPiDay was used for the "Saturday Night On-line" show promotion and as a news occasion.\\


   \begin{figure}[!hbt]
            \begin{center}
            \includegraphics[width=\columnwidth]{hillary_ratio_top5.png}
            \caption{Word cloud of top hashtags with ratio bigger than or equal to 5 in \#Hillary2016 dataset.}
            \label{fig:word_hillary}
            \end{center}
      \end{figure}
    

In the dataset for \#Hillary2016, \#ohhilno has degree \= 3887 and ratio \= 1943, which is significantly bigger than others. This hashtag is posted by a single user @MikeVee5 in an automated manner via an application. The overall frequency of tweets for this user is high with a high number of repeated content. Tweets are related to politics and most of them contain messages and hashtags against Hillary Clinton candidature as well as links to the political web-site "The Hill Talk"~\cite{thehilltalk}. This user may be identified as attempting to influence opinion. \\ 

Another interesting observation for the \#ohhilno hashtag is that it is semantically close to the hashtag \#ohhillno, which is used by a bigger number of users and have a degree equal to 113 other hashtaghs it is connected to against degree of 2 for \#ohhilno. This is caused by the fact that Twitter provides suggestions for hashtags written in tweets manually. All the users, that were going to post tweets and had the intention to add this hashtag received the suggestion of \#ohhillno hashtag. Hence, the behavior of \#ohhillno is usual, but the behavior of \#ohhilno is suspicious. \\

In the set of hashtags with the ratio $>=$ 5 we found a subset of hashtags posted by a set of four users. The hashtags are: conspiracy (degree = 170), murder (degree = 63), lewinsky (degree = 31), sex (degree = 10). These hashtags are used in separate tweets with different content but tweeted by accounts which represent ideas against animal euthanasia on the American farms. Tweets are posted in automated manner and may be related to opinion promotion campaigns and spam. \\

% What do you mean by this sentence?
% For the data related to the \#Hillary2016 hashtag most of the tweets which contain suspicious hashtags have the goal to influence their audience for or against Hillary Clinton. Number of tweets with spam, which would be unrelated o the topic is very small.

To see if our method will identify the actual spam campaigns, we have chosen another hashtag \#iPhone for which we collected data with a significant number of spam tweets. Data wes collected for the following periods: May 9th - May 11th; May 23rd - May 25th; June 1st. Total number of collected tweets is 97315. We computed the frequency and degree of each hashtag and ratio between them. The result is presented in Figure~\ref{fig:scatter_ratio_iphone}.
\\
        \begin{figure}[!hbt]
            \begin{center}
            \includegraphics[width=\columnwidth]{iphone_scatter_ratio_without_retweets.png}
            \caption{Degree \& frequency scatterplot and ratio for \#iPhone.}
            \label{fig:scatter_ratio_iphone}
            \end{center}
      \end{figure}
    
In figure~\ref{fig:scatter_ratio_iphone} we observe a significant number of hashtags with high ratio. For the purposes of our analysis we have chosen the same threshold as for the two previous campaigns. Table~\ref{tab:hashtagratioiphone} provides the detailed information about the observed behavior of frequency-degree ratio. Number of tweets with ratio $>=$ 5 is bigger than the same number for previous campaign. This gives us an intuition that in \#iPhone dataset we can find more suspicious hashtags and tweets posted by users with unusual activity. Among these hastags the number of suspicious hashtags is also bigger reaching the value of 46,2\%.
\\
    
     \begin{table}[!hbt]
            \begin{center}
            \caption{Analysis of top hashtags for \#iPhone with ratio $>=$ 5}
            \label{tab:hashtagratioiphone}
            \begin{tabular}{|p{5cm}|p{3cm}|}
                  \hline
                   & \#iPhone \\
            \hline
                  Total number of hashtags & 5938 \\
                  \hline
                  Number of hashtags with ratio $>=$ 5 & 201 \\
                  \hline
                  \% of hashtags with ratio $>=$ 5 & 3,38\% \\
            \hline
            Number of suspicious hashtags & 93 \\
                  \hline
            \% of suspicious hashtags & 46,2\% \\
                  \hline
            \end{tabular}
            \end{center}
      \end{table}
    
    
     \begin{figure}[!hbt]
            \begin{center}
            \includegraphics[width=\columnwidth]{iphone_ratio_top5.png}
            \caption{Word cloud of top hashtags with ratio bigger than or equal to 5 in \#iPhone data set}
            \label{fig:word_iphone}
            \end{center}
      \end{figure}
    
    
The top hashtags with the degree:frequency ratio $>=$ 5 form the word cloud shown in Figure~\ref{fig:word_iphone}. For a more precise analysis we have chosen ten hashtags from the set of hashtags with ratio $>=$ 5. These hashtags along with related frequency, degree and ratio are presented in table~\ref{tab:hashtagratioiphonetop}.
\\

 \begin{table}[!hbt]
            \begin{center}
            \caption{Top 10 hashtags for \#iPhone with the highest ratio (ordered by ratio).}
            \label{tab:hashtagratioiphonetop}
            \begin{tabular}{|p{5cm}|p{2cm}|p{2cm}|p{2cm}|}
                  \hline
                  Hashtag & Frequency & Degree & Ratio \\
            \hline
                  1. Movie & 11142 & 4 & 2785 \\
                  \hline
                  2. GameInsight & 14599 & 13 & 1123 \\
                  \hline
                  3. Sex & 11187 & 10 & 1118 \\
            \hline
            4. iPhoneGames & 14808 & 30 & 493 \\
                  \hline
            5. Hot & 11153 & 32 & 348 \\
            \hline
            6. Bigdays & 312 & 1 & 312 \\
                  \hline
            7. Analytics & 1031 & 4 & 257 \\
            \hline
            8. FishEye & 1165 & 5 & 233 \\
            \hline
            9. Lomo & 1165 & 5 & 233 \\
            \hline
            10. Cellular & 187 & 1 & 187 \\
            \hline
            \end{tabular}
            \end{center}
      \end{table}
    
    
Hashtags 1, 3, 5 have similar meaning, and appear in the same tweets. The set of users posting tweets is almost the same, with minor changes. There are 160 users posting inappropriate tweets with these hashtags. The usernames have the same pattern, most of them contain inappropriate words. All tweets that contain hashtags 1, 3 and 5 may be considered as spam with abusive and adult content. \\

Hashtag 2, \#GameInsight, is related to a company ``Game Insight''. ``Game Insight is a global developer and publisher of free-to-play mobile games and social games.''~\cite{wiki_gameinsight}. Tweets with this hashtag are posted by 1319 users in automated manner with different frequency. The game application is linked to a Twitter account of a user and is posting tweets on users behalf depending on users progress in the game. \\

Hashtag 4, \#iPhoneGames, was posted by 1322 users in the same manner as the \#GameInsight. 98\% of users posted tweets with \#iPhoneGames are the same. So we conclude that these two hashtags are strongly connected and appear in a couple. \\

Hashtag 6, \#Bigdays, is posted by accounts that belong to British on-line coupon service ``Big Days''~\cite{big_days}. All the tweets with \#Bigdays are used for promotion and advertisement of the company itself as well as its services. \\

Hashtag 7, \#Analytics, appear mostly with the following set of hashtags [``Software'', ``iPhone'', ``app'', ``Apple'', ``appleipad''] and is posted by a set of 308 users. We looked through top 30 users which posted this set of hashtags bigger number of times, and 27 of them were suspended (as for 11th June 2015) and 3 users were not found, due to changed username or the fact that they might be completely deleted. The notion of suspended account is defined by Twitter as blocking an account which violated ``The Twitter Rules''~\cite{twitter_rules}. Most of these 308 accounts have similar usernames, with added keyword ``news'' at the end or in the middle of a username. This gives an impression of serialized accounts created on purpose of spamming activities. According to ``The Twitter Rules'': \textit{``Mass account creation may result in suspension of all related accounts.''}~\cite{twitter_rules}. We assume that this was the reason these accounts were suspended.
\\

Hashtags 8, \#FishEye, and 9, \#Lomo, have the same degree and frequency by this we can conclude that they have a high probability to be connected. These two hashtags appear together in 1165 tweets posted by the same set of 25 accounts. These hashtag are connected to a set of hashtags [``app'', ``ios'', ``appstore'', ``game'', ``iphone''] which appears in every tweets with \#Lomo and \#FishEye. These five hashtags have a more general meaning, hence they appear in the bigger number of tweets than \#Lomo and \#FishEye and are connected to a bigger number of tweets. Nevertheless, \#Lomo and \#FishEye couple allows us to identify unusual behavior of 25 accounts. Among these accounts there is a set of accounts that belong to an application LUZMO~\cite{itunes_luzmo} that posts tweets automatically and may be considered as an organized promotion campaign. Other users in the set also have an automated behavior due to applications linked to their accounts which post tweets on users behalf.
\\

Hashtag 10, \#Cellular, is posted in a couple with \#Deals by one user @homeshopbuzz~\cite{homeshopbuzz}. The account is used for the promotion and advertisement of different kind of goods. Each tweet contains a link to an on-line shop ``Ebay'', posted automatically through linked application.
\\
 
We can conclude that our method is able to identify hashtags which are mostly used by suspicious users for different kind of automated activities. Depending on the character of the collected data, we can recognize accounts used for opinion manipulation, promotion, advertisement or spam.

% \section{Contributions}


% I think contribution refers to the work that the project has acheived and not what indivdual people did.


% Anuvabh was responsible for the data collection and analysis. Anastasia was responsible for data analysis and result representation. Intermediate and final report were divided.
      

\section{Conclusion}
Our work studies the impact of influence of users on their followers and the relations between tweets. We study how users with a large audience impact the tweeting behavior of their followers. We conclude that they result in an impulse where there is a increase of tweets for a very short time period. Popular users do not significantly alter the behavior of other Twitter users. We also observe that the number of original posts of users stay constant and they are more than the retweets of these popular users. \\

We take a different approach in studying the behavior of hashtags. We choose hashtag campaigns and study how users share information using such a hashtag. We discover that spammers try to piggyback on popular hashtags by including the popular hashtag along with their content. We uncover a large group of coordinated accounts that use the same set of hashtags for tweeting. We finally identify users having suspicious activity by studying the created graph. We see that hashtags which have low connections to the rest of the graph are used by these suspicious users. We are also able to identify some of the accounts that use these hashtags as users with unusual activity.

\section{Future Work}

We plan to extend our work to general hashtag campaigns. For now we have carried out our analysis on data from specific hashtag campaigns. It would be interesting to see the results on data from several hashtag campaigns. During the project we collected tweets from Twitter's streaming API and we will apply our analysis to it. We will continue to work on the project during the summer as interns at INRIA. \\

Another area that we can work on is using a sentiment analysis engine to increase the amount of automation in the system. This is would help as a preliminary classification phase. Combining this with the application of the technique on general datasets, we may try to build a real time system that can identify when opinion manipulation campaigns are being started by users.

\begin{thebibliography}{10}

      \bibitem{twitter_stats}
    Official Twitter statistics on the company website.
    \url{https://about.twitter.com/company}, accessed 12th June, 2015.
    
      \bibitem{wikihashtag}
    Wikipedia article about ``Hashtag''.
    \url{http://en.wikipedia.org/wiki/Hashtag}, accessed 11th June, 2015.
      
    \bibitem{thomas_paper}
    Kurt Thomas, Chris Grier, Dawn Song, and Vern Paxson. ``Suspended accounts in    retrospect: an analysis of twitter spam''. In \textit{IMC '11 Proceedings of the 2011 ACM SIGCOMM conference on Internet measurement conference.} Pages 243-258 
%     \url{http://dl.acm.org/citation.cfm?id=2068840}
    
    \bibitem{benevenuto_paper}
    Fabricio Benevenuto, Gabriel Magno, Tiago Rodrigues, Virgílio Almeida. ``Detecting spammers on twitter''. In \textit{Collaboration, Electronic messaging, Anti-Abuse and Spam Conference (CEAS), 2011}

      \bibitem{thehilltalk}
    The link to a political web-site ``The Hill Talk''.
    \url{http://thehilltalk.com/}, accessed 10th June, 2015.

      \bibitem{dragplus_audio}
    The link to a web-site ``Dragplus'' with advertising content.
      \url{http://dragplus.com/page/audio}, accessed 10th June, 2015.

      \bibitem{big_days}
    The link to an official site of the ``Big Days'' company.
      \url{http://www.bigdays.co.uk/}, accessed 10th June, 2015.

      \bibitem{wiki_gameinsight}
      Wikipedia article about ``Game Insight'' company.
    \url{http://en.wikipedia.org/wiki/Game_Insight}, accessed 10th June, 2015.
    
    \bibitem{twitter_rules}
    Official Twitter rules for users.
    \url{https://support.twitter.com/articles/18311-the-twitter-rules}
    
    \bibitem{itunes_luzmo}
      The link to LUZMO (Light Effects Powerful Photo Editor Blend Light Leaks Textures and Overlays) application on iTunes.
    \url{https://itunes.apple.com/th/app/luzmo/id681205606?mt=8}, accessed 10th June, 2015.
    
    \bibitem{homeshopbuzz}
    The link to the Twitter account of @homeshopbuzz user, accessed 11th June, 2015.
    \url{https://twitter.com/homeshopbuzz}

\end{thebibliography}

% Your document ends here!
\end{document}